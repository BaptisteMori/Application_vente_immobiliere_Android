\documentclass[a4paper,12pt]{article} %style de document
\usepackage[utf8]{inputenc} %encodage des caractères
\usepackage[french]{babel} %paquet de langue français
\usepackage[T1]{fontenc} %encodage de la police
\usepackage[top=2cm,bottom=2cm,left=2cm,right=2cm]{geometry} %marges
\usepackage{graphicx} %affichage des images
\usepackage{amssymb}
\usepackage{url}
\usepackage{verbatim}
\usepackage{amsmath}
\usepackage{color}
\usepackage{tikz}
\usepackage{hyperref}
\hypersetup{
	hidelinks,
    colorlinks,
    citecolor=black,
    filecolor=black,
    linkcolor=black,
    urlcolor=black
}

\begin{document} %début du document

%----------------------------------
%page de garde
%----------------------------------

\begin{titlepage}

\vspace{7cm}

\begin{center}

\begin{Huge}
Rapport Android\\
\end{Huge}
\vspace{2cm}
\begin{large}
Chagneux Dimitri 21606807\\
Mori Baptiste 21602052\\
Leblond Valentin 21609038\\
\vspace{1cm}
Lien du projet:\\
\url{https://github.com/BaptisteMori/Application_vente_immobiliere_Android}\\
\vspace{1cm}
L3-Informatique
\end{large}

\end{center}
\end{titlepage}


%------------------------------
%sommaire
%------------------------------

\newpage

\tableofcontents

\newpage

%------------------------------
%contenu
%------------------------------


\section{Présentation du sujet}

L'objectif de ce devoir était de réaliser une application de vente de biens immobiliers.

\section{Fonctionnalités implémentées}

\subsection{Liste de biens}
Du côté de la récupération des annonces en ligne, nous avons commencé par créer une requête pour aller chercher le fichier en ligne.
A partir de la le résultat obtenu est un fichier au format json , ce qui implique l'utilisation de la librairie Moshi pour parser ce fichier. \\
Pour cela nous avons créé dans un premier temps un parseur pour traiter le cas où il n'y a qu'une seul propriété puis un second parseur pour le cas où il y a plusieurs annonces.\\
Nous avons était obligé de gérer ces deux cas car dans le cas où il y a plusieurs annonces celle ci sont dans une liste contrairement au cas où il n'y en a qu'une seul.

\subsection{Base de donnée local}
Le titre dit tout...

\subsection{Visualisation d'un bien}
La visualisation d'une propriété se fait sur une activité appelée ActivityAnnonce.
On y affiche le titre, une galerie d'images, le prix et la localisation, une description et des informations sur le vendeur. On ajoute en plus, si la propriété est sauvegardée dans la base de donnée locale, une liste de remarques.

\subsection{Menus}
Notre application dispose de deux menus: le premier est celui qui se trouve sur la première activité, lorsqu'on démarre l'application. Il permet d'accéder à la liste des propriétés sauvegardées en local. Le second est celui de l'activité qui affiche une annonce. On y trouve:\\
    -un bouton pour aller sur la liste de toutes les annonces (menu);\\
    -un bouton pour prendre une photo;\\
    -un bouton permettant d'enregistrer l'annonce en local, qui affiche un message dans une snackbar si l'annonce est déjà présente dans la base de donnée;\\
    -un bouton pour supprimer l'annonce de la base de donnée locale, qui affiche également un message dans une snackbar si l'annonce n'est pas dans la base;\\
    -un bouton listant les annonces sauvegardées en local (de la même manière que dans le premier menu);\\
    -un bouton pour ajouter une remarque si l'annonce se trouve en local. Dans le cas contraire on affiche un message dans une snackbar.

\subsection{Remarques et photos}
En ce qui concerne les remarques, il est possible d'en ajouter si la propriété se trouve sur la base de donnée en local. Si c'est le cas, en cliquant sur le bouton du menu d'une annonce on arrive sur une nouvelle vue contenant un champ de texte, un bouton retour et un bouton valider.
Au clique sur le bouton valider, la remarque est ajoutée en base de donnée et s'affiche sur la vue de l'annonce concernée. Elle est supprimée si l'annonce est supprimée de la base de données locale.
\vspace{1cm}

Pour les photos, l'appui sur l'icône de l'appareil photo ouvre l'application caméra par défaut, seulement si l'annonce est sauvegardée en locale.\\
Une fois la photo prise, on l'enregistre dans les fichiers de l'application avec un nom du type JPEG-yyyyMMdd-HHmmss.jpg". On ajoute ensuite le chemin de l'image dans la liste des images de l'application.

\section{Gestion du projet}

\subsection{Organisation}
Commun:\\
-Débuts de l'application (vue d'une annonce, lister des annonces codées en dur)
\vspace{0.5cm}

Baptiste:\\
-Moshi\\
-Aide sur les autres aspects
\vspace{0.5cm}

Valentin:\\
-Base de données
-Remarques (base de données)\\
-Aide sur les autres aspects\\
-Aspects graphiques (liste des annonces)
\vspace{0.5cm}

Dimitri:\\
-Liste des annonces, à partir du Moshi, dans une vue\\
-Remarques\\
-Prise de photos

\subsection{Difficultés}
Moshi, prise de photo (affichage dans la galerie surtout)...

\end{document}
